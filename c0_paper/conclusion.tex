\section*{Conclusion}

In conclusion, we devise a new architecture for $C^0$ continuous interpolation, and demonstrate
that it works on Dynamic Neural Rendering. Our architecture is able to accurately and quickly
reconstruct scenes, while providing stronger guarantees with a well understood tool.
Hopefully, this inspires more introduction of classical tools inside of the differentiable rendering pipeline.

While our work is incremental, requiring very little code, because it changes the underlying structure there are a significant number of extensions that can be done with it.

\section*{Future Work}

For future work, we plan on applying Bezier splines to other signals, such as sound for
reconstruction of long music. Sound must be continuous, even if it has sharp changes, and may change significantly over time, so our architecture is suitable for its reconstruction.

Another next step in neural rendering is to encode this using a Plenoxel~\cite{yu2021plenoxels} or another sparse structure, to allow for rapid training of dynamic scenes. To the author's knowledge, there is no real-time construction of 3D scenes since there did not exist a deep-learning approach to reconstructing movement. Bezier splines fill in this gap, and since they only require a fixed number of parameters, this should allow for efficient rendering and training without requiring an MLP evaluation. If this follows the trend of scene reconstruction, it may be on the order of 100 times faster to reconstruct dynamic scenes, without any loss in quality, and we hope that our work gets adopted for this purpose.

It may also be interesting to explore different formulations of splines, as we only select
Bezier splines due to ease of implementation, but it may be that other splines might be more
numerically stable or have stronger expressivity for reconstruction.

\section*{Acknowledgements}

Thanks to Elliot Cuzzillo for helping proofread this work.