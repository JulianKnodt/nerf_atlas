\section*{Conclusion}

In conclusion, we devise a new architecture for $C^0$ continuous interpolation, and demonstrate
its efficacy on Dynamic Neural Rendering. Our architecture is able to accurately and quickly
reconstruct scenes, while providing stronger guarantees with a simpler formulation than before.
This hopefully inspires more introduction of classical tools inside of the differentiable rendering pipeline.

\section*{Future Work}

For future work, we plan on applying this architecture to other signals, such as sound for
reconstruction of long duration music. This can be used to generalize over sound, by generating
an embedding from using an encoder-decoder architecture.

Another next step, in the realm of neural rendering, is to encode this using a Plenoxel, so that
rapid training can be done. Since we are storing a fixed number of parameters predicted from a
neural network, this should allow for much more efficient rendering and training, on the order
of 100 times faster, but make it significantly easier to reconstruct dynamic scenes. We note that this is now possible with our formulation, because it is not necessary to query a neural net if the spline is already known, and this can be cached in a sparse voxel structure.

It may also be interesting to explore different formulations of splines, as we only select
Bezier splines due to ease of implementation, but it may be that other splines might be more
numerically stable or have stronger expressivity for reconstruction.
