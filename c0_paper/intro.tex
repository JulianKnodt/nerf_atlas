\section*{Introduction}

The problem of learning smooth functions is a key problem in machine learning, as many problems
are framed as finding continuous solutions to problems with only a small number of data points.
This smoothness is often not guaranteed, and is enforced through regularization of the output,
usually using something like $L_2$ loss or temporal consistency~\cite{tretschk2021nonrigid}. Since
the consistency is usually imposed as some weighted loss, it is not a guaranteed property of the
network. This leads to difficulties in interpolation between time-steps, which can be used to
exaggerate motion, or predict intermediate time steps. In addition, there is also the
possibility for sudden changes in motion.
Notably, we are interested in constructing functions with $C^0$ continuity. That is, the
function must be continuous on some domain. We are also to some extent interested in $C^1$
continuity, or the derivative of a function is continuous on some domain, which often is
reflected in the realism of motion by enforcing smooth velocity changes.

We examine the specific problem of predicting movement for dynamic scenes using NeRF. For this,
we define some static configuration, which is referred to as the canonical scene. From this, we
march rays into the scene, warping them in order to deform the rendered view, which is
essentially a perspective shift from moving objects in the scene. This is layered on top of the
original NeRF model in order to reconstruct movement from a fixed set of views. This simple
formulation is highly effective at reconstructing lambertian scenes with continuous movement as
described in D-NeRF~\cite{pumarola2020dnerf} and NR(non-rigid)-NeRF~\cite{tretschk2021nonrigid}. These works show convincing
reconstructions of dynamic movement in scenes, predicting accurate reconstructions on both real
and synthetic scenes. If we are satisfied with pure reconstruction these methods suffice, but we
are also interested in performing operations on movement, such as interpolating between frames,
exaggerating it, or classifying regions of movement. These methods demonstrate capability to
perform well on certain tasks such as motion interpolation, but require specific optimizations
in order to demonstrate improvements. In contrast, we examine classical animation tools in order
to better handle movement.

In animation, there are many existing tools for creating smooth movement with a high-degree of
control for artists and animators. For example, there are tools such as keyframing, splines,
and other techniques. These tools allow artists and animators to breathe life into animation,
with a high degree of control. Mathematical constructs such as splines have also been thoroughly
studied to understand their behaviour, as well as how they can be manipulated. These tools are
clearly powerful enough to represent most movement, but have not been leveraged in machine
learning for this purpose.

We blend these existing tools in order to apply a prior on movement in Dynamic NeRF, finding that we are able to get a slight increase in performance without additional cost. In addition, we propose a new architecture for long duration $C^0$-continuous signal reconstruction. In summary, our contributions are as follows:

\begin{enumerate}
    \item A formulation of a prior on top of existing D-NeRF, which enforces smoothness of movement, a canonical basis, and an analytic form at no additional cost, while slightly outperforming the original.
    \item An evaluation of our architecture against D-NeRF which demonstrates its performance.
    \item A general architecture for reconstructing long, continuous signals, which guarantees smoothness. This architecture can be used to reconstruct long, dynamic sequences, but is more general and can be applied broadly to smooth signal reconstruction.
\end{enumerate}
