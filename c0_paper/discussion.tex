\section*{Discussion}

Our method produces coherent movement in objects without explicit regularization, and is able to
converge on the same order as compared to prior work. It also does not suffer extremely significant deformations around time extremities as compared to D-NeRF, since we are analytically able to ensure that at time 0 there is no motion.

In addition, our method demonstrates extremely smooth interpolation, as enforced by the structure of our network.
We are thus able to exaggerate movements significantly, while ensuring plausibility between frames. In addition, because of the analytic nature
of splines, we are likely able to perform interesting motion deformations, such as diverging motion paths in order to construct novel motion.

One issue with our approach is that because it is forced to learn continuous representations, it has some difficulty when surfaces collide and bounce apart, as it must separate each surface but the boundary between them may be very small and hard to distinguish. We expect that in the limit these issues would vanish, but they slow down training.

Another component of our model are the computational challenges with spline evaluation. We are currently unaware of a fully vectorized method for evaluating splines at a given point, and developing an algorithm for such would benefit our model.