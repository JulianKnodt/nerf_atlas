\section*{Discussion}

Learned Bezier splines produce coherent movement in objects without explicit regularization, and are able to converge with equal to speed as prior work. Our method demonstrates smooth interpolation, as enforced by the structure of our approach. We are thus able to super-sample between frames at arbitrary resolution while guaranteeing smoothness. In addition, because of the analytic nature of splines, we could perform interesting motion deformations, such as twisting the curves in order to construct novel motion.

Another computational benefit of our approach is that for a given view, we only need to evaluate the deformation MLP \textit{once} to compute the Bezier spline at each point, then we can sample the Bezier spline at an arbitrary number of points. This should allow for efficient single view video reconstruction, but we leave future work for this.

Finally, even without the above optimization, our model runs approximately as fast as NR-NeRF, but can be optimized further, either by introducing voxels, or using other learning approaches such as a different optimization scheme for splines.