\documentclass[10pt,twocolumn,letterpaper]{article}

\usepackage{cvpr}
\usepackage[colorlinks=true,linkcolor = blue,urlcolor = blue]{hyperref}
\usepackage{amsmath}
\usepackage{amssymb}
\usepackage{graphicx}
\usepackage{siunitx}
\usepackage[table]{xcolor}


%\cvprfinalcopy

\begin{document}


\title{$C^0$-Continuous Networks for Smooth Interpolation}
\author{Julian Knodt}
\affiliation{
  \institution{Princeton University}
  \city{Princeton}
  \state{New Jersey}
  \country{USA}
}

\maketitle

\begin{abstract}
Methodology for predicting smooth functions over time using neural nets has previously relied on
some form of consistency loss to ensure smoothness over time, which has showed good results on
short sequences, where it is able to effectively apply the loss function over the whole duration.
This approach is difficult to scale to larger sequences, as it requires more pairwise
constraints as the sequence length grows.

We propose an alternate formulation, which ensures $C^0$ continuity of sequences using a
Bezier-spline formulation. We examine the problem of smooth function reconstruction for dynamic
scenes using NeRF, but note that our approach is much more general and can be applied to a large
variety of problems. For dynamic NeRF, we are interested in reconstructing a ray-warping
function $f(x\in\mathbb{R}^3,t\in[0,1])\to\delta(x)$. We decompose this into two functions,
$f(x\in\mathbb{R}^3)\to\beta, B_\beta(t)\to\delta(x)$, where $B_\beta$ is a Bezier spline
parametrized by the control points $\beta$. This allows for $C^0$, and in this case $C^1$ continuity,
allowing for perfect interpolation in time. We model $f(x\in\mathbb{R}^3)\to\beta$ as an MLP,
blending machine learning with classical animation techniques. All code is available at \url{https://github.com/JulianKnodt/nerf_atlas}.
\end{abstract}

\begin{figure*}
    \centering
    \includegraphics[width=\textwidth]{teaser}
    \caption{The proposed model is able to recover dynamic movements in synthetic data, with plausible flow maps. With a classical formulation of movement using Bezier splines, we impose a stronger prior on the kind of movement that can be reproduced, while enforcing that it is smooth. We compare our model to an implementation of D-NeRF~\cite{pumarola2020dnerf}, and find our model is able to slightly outperform it. This can be seen qualitatively in video, as the movement characterized by ours is much smoother.}
    \label{fig:intro_figure}
\end{figure*}

\section{Introduction}

The problem of learning smooth functions is a key problem in machine learning, as many problems
are framed as finding continuous solutions to problems with only a small number of data points.
This smoothness is often not guaranteed, and is enforced through regularization of the output,
usually using something like $L_2$ loss or temporal consistency~\cite{nr-nerf}. Since
the consistency is usually imposed as some weighted loss, it is not a guaranteed property of the
network. This leads to difficulties in interpolation between time-steps, which can be used to
exaggerate motion, or predict intermediate time steps. In addition, there is also the
possibility for sudden changes in motion.
Notably, we are interested in constructing functions with $C^0$ continuity. That is, the
function must be continuous on some domain. We are also to some extent interested in $C^1$
continuity, or the derivative of a function is continuous on some domain, which often is
reflected in the realism of motion by enforcing smooth velocity changes.

We examine the specific problem of predicting movement for dynamic scenes using NeRF. For this,
we define some static configuration, which is referred to as the canonical scene. From this, we
march rays into the scene, warping them in order to deform the rendered view, which is
essentially a perspective shift from moving objects in the scene. This is layered on top of the
original NeRF model in order to reconstruct movement from a fixed set of views. This simple
formulation is highly effective at reconstructing lambertian scenes with continuous movement as
described in D-NeRF~\cite{dnerf} and NR(non-rigid)-NeRF~\cite{nr-nerf}. These works show convincing
reconstructions of dynamic movement in scenes, predicting accurate reconstructions on both real
and synthetic scenes. If we are satisfied with pure reconstruction these methods suffice, but we
are also interested in performing operations on movement, such as interpolating between frames,
exaggerating it, or classifying regions of movement. These methods demonstrate capability to
perform well on certain tasks such as motion interpolation, but require specific optimizations
in order to demonstrate improvements. In contrast, we examine classical animation tools in order
to better handle movement.

In animation, there are many existing tools for creating smooth movement with a high-degree of
control for artists and animators. For example, there are tools such as keyframing, splines,
and other techniques. These tools allow artists and animators to breathe life into animation,
with a high degree of control. Mathematical constructs such as splines have also been thoroughly
studied to understand their behaviour, as well as how they can be manipulated. These tools are
clearly powerful enough to represent most movement, but have not been leveraged in machine
learning for this purpose.


\section{Background}

\subsection{NeRF}

TODO put in some reference to NeRF here.

\subsection{Bezier Curves}

Bezier curves refer to a class of functions defined as a polynomial parametrized by a set of
control points. They are most commonly used as cubic polynomials: $f(x) = ax^3 + bx^2 + cx + d$,
where x is the variable we are interested in interpolating over. The general formulation for
the Bezier basis functions is defined as $B^n(t) = \Sigma^n_{i=0} {n}\choose{i} (1-t)^{n-i} t^i$ where n
is the degree of the Bezier polynomial. In order to give control of the Bezier curve, we
introduce "control points", which weighs different points along the curve differently:
$B^n(t) = \Sigma^n_{i=0} P_i {n}\choose{i} (1-t)^{n-i} t^i$, where $P_i\in\mathbb{R}^3$ for 3D
movement. For a more complete reference on Bezier splines, we refer the reader to a more
\href{https://pomax.github.io/bezierinfo/index.html}{complete reference}.



\section*{Method}

Our method consists of imposing additional structure on top of existing machine learning
approaches to reconstruction. For some function $f(p, t)\to\mathbb{R}^n, t\in[0,1]$ which is modelled through
some learned approach, we decompose it into a function $f(p)\to\mathbb{R}^{n\times O}, B_O(f(p), t)$, where
$O$ is the order of the Bezier spline, $f$ is the learned control points, and $B_O$ is the evaluation
of the $O$'th order Bezier spline with control points defined by $f(p)$.

\subsection*{Architecture}

Specifically for the case of dynamic NeRF, we formulate $f(p)$ as
$\text{MLP}(x,y,z)\to\mathbb{R}^{3\times O}$. We raymarch from a camera with known position and
view direction through the scene, and at every point we compute the set of control points for
the amount the ray is being bent. We then are able to evaluate it at any time $t=[0,1]$ in order
to reconstruct movement, $\Delta_0(x)$. The number of spline points is variable, but we are able
to reconstruct movement with as low as 4 spline points, but experiment with at least 16 spline
points. In order to evaluate the Bezier curve in a numerically stable way, we use De Casteljau's
algorithm, and further plan on extending it to handle rational bezier splines, which would allow
for even greater control of movement.

De Casteljau's algorithm evaluated at time $t$ is characterized by the recurrence relation:
\[ \beta_i^{(0)} = \beta_i \]
\[ \beta_i^{j} = (1-t)\beta_i^{(j-1)} + t\beta_{i-1}^{(j-1)} \]
Which can be conceptually thought of as linearly interpolating between adjacent control points until there is only a single point remaining. This takes $O(n^2)$ operations to evaluate, where $n$ is the number of control points.

In addition, we are also interested in constructing a reasonable canonical NeRF, and arbitrarily select $t = 0$ to be the canonical NeRF. Thus, we are interested in Bezier Curves where $B_O(0) = \overrightarrow{0}$. This can be achieved in two different ways, either by assigning $p_0 = \overrightarrow{0}$, and only compute $f(p) = p_{1\cdots O-1}$. Then, we can use the Bezier spline with the control points as $[\overrightarrow{0}, p_{1\cdots O-1}]$. Alternatively, we can compute $p_{0\cdots O-1}$ and use the Bezier spline with control points $[p_{0\cdots O-1}]-p_0$, and the final change in position is $B_O(f(p)-f(p)_0,t)+p_0$. While both formulations are in theory equivalent, and explicitly concatenating 0 leads to fewer degrees of freedom, we find the second one to lead to better convergence near 0, so we use that in our models.

Following NR-NeRF, we also apply a rigidity to every ray, which is also computed as a function of position:
\[ \text{Rigidity}\in[\epsilon, 1] =\sigma_{\epsilon^\uparrow}(\text{MLP}(x,y,z)) \]
\[ \sigma_{\epsilon^\uparrow}(v) = \sigma(v)(1-\epsilon) + \epsilon \]
Which essentially rescales the
difficulty of learning movement, making it more easy to represent and classify static scene
objects. The final change in position is thus defined as $\Delta(x) = \text{Rigidity}\times
\Delta_0(x)$.

In order to reconstruct RGB values, we also diverge from traditional NeRF, and only allow for
positional dependent colors:
\[ \text{RGB} = \sigma_{\epsilon^\Leftrightarrow}(\text{MLP}(x,y,z)) \]
\[ \sigma_{\epsilon^\Leftrightarrow}(v) = \sigma(v)(1+2\epsilon) - \epsilon \]
Using the
activation function as described in MIP-NeRF~\cite{barron2021mipnerf} in order to better reconstruct colors. Because of the low number of
samples for a moving object at a given view, it is much more difficult to learn specular
reflection, thus it becomes necessary to only model the Lambertian component of the color. This is in line with NR-NeRF and D-NeRF, which models the diffuse component.

\subsection*{Loss}
While training, we also introduce an additional loss term, as the $\ell_2$ loss may have
difficulty when there is a large pixel-wise gap in movement between the learned and predicted
component. We formulate the loss term as $\text{MSE}(\text{FFT}(I_\text{GT}),
\text{FFT}(I_\text{predicted}))$, where the FFT is the 2D fast fourier transform of an image.
The MSE here is defined over the complex domain as the FFT has both a real and imaginary
component, and is defined as \[ L_\text{FFT} =\text{MSE}(a\in\mathbb{C}^k,b\in\mathbb{C}^k) \]
\[ = \frac{1}{k}\Sigma_{i=0}^k|a_i-b_i| \]
\[ |z\in\mathbb{C}| = (\Re(z)+\Im(z))(\Re(z)-\Im(z)) \]
We introduce this term as a cheap replacement to structural similarity, since the FFT captures
information about the structure of the whole image, which is useful in a dynamic setting due to
the disjointedness of where a predicted object may be as opposed to its final position\footnote{In a published iteration, we would likely perform an ablation of this loss function.}.

Our final loss formulation is thus:
\[ \ell_2(I_\text{predicted}, I_\text{GT}) + L_\text{FFT}(I_\text{predicted}, I_\text{GT}) \]
without any additional regularization terms.

Our formulation is simpler compared to NR-NeRF since there is no need to regularize over
temporal consistency, and that makes the optimization process simpler. Our training process also does not require adding in additional frames over time for relatively short sequences, so we randomly sample from the set of frames. This is contrast to D-NeRF, which requires adding frames in during training since it has to enforce that at $t=0$ the predicted movement is 0.

\subsection*{Training}

For training, we sample random crops of given frames, computing L2 and FFT loss and back-propagating through the whole network. We use autograd to optimize the splines, but note that there are also classical approaches to solving them which may lead to faster optimization in the future. We use the Adam optimizer~\cite{Kingma2015AdamAM} with cosine simulated annealing~\cite{loshchilov2017sgdr} to go from $\num{2e-4}$ to $\num{5e-5}$ over the course of a day.

In order to ensure that our model is robust to small changes in camera angle, we introduce subpixel jitter in the camera, identical to NeRF. We dot not introduce any jitter in the time-domain, since we are guaranteed smoothness from our formulation.

\subsection*{Long Duration $C^0$ Continuity}

While the above structure is sufficient to model short-term $C^0$ continuity, it runs into
computational instability and higher cost the longer the sequence is, due to the $O(n^2)$
evaluation cost of De Casteljau's algorithm. We also design an additional architecture which
composes the previous architecture for many small $C^0$ curves. Fundamentally, this idea permits
for reconstruction of infinitely long sequences with guaranteed smoothness at little extra cost.

We define the architecture akin to poly-splines, or a composition of many small Bezier splines.
For a known-length sequence, we divide it into $K$ different segments, and without loss of
generality assume that $K$ evenly divides the total number of frames. This implies that $t$ is
now in the range $[0, K]$, and can be decomposed into a segment number and fractional component:
\[ k\in\mathbb{Z}_K=\lfloor t\rfloor, t'\in[0,1)=t-k \]
Then we define an embedding $\text{Emb} = \mathbb{R}^{Z\times(K+1)}$, where $Z$ is some latent
dimension size. We then create an "anchor" network:
\[
    \textit{anchor}(x\in\mathbb{R}^3,z\in\mathbb{R}^Z) = \text{MLP}(x,z)\to(\mathbb{R}^3,\mathbb{R}^{Z'})
\]
So named because it computes "anchors"
or endpoints of the curves, as well as a representative latent vector. Between these two
anchors, we are interested in computing some number of intermediate spline points. We define the
control point estimation network as
\[
\textit{control}(x\in\mathbb{R}^3, z_1, z_2\in\mathbb{R}^{Z'}) = \text{MLP}(x,z_1,z_2)\to\mathbb{R}^{O\times\mathbb{R}^3}
\]
Where $z_1,z_2$ are the
representative latent vectors from the anchor network, and it outputs $O$ spline points in
$R^3$. We optionally also can include a per-segment rigidity value, similar to above, but lose
some guarantees of continuity by doing so.

In order to evaluate the spline at a given time $t$ for a point $x$, we first compute $k, t'$. Then, we compute
the anchor points at
\[ p_0,z_1=\textit{anchor}(x,\text{Emb}[k]) \]
\[ p_\text{end},z_2=\textit{anchor}(x,\text{Emb}[k+1]) \]
And the middle control points as \[ p_{1\cdots O-2} = \text{control}(x,z_1,z_2) \]
The final set of control points we use
is the concatenation of the anchors with the intermediate points:
$[p_0, p_{1\cdots O-2}, p_\text{end}]$. Using this set of control points, evaluate the Bezier
curve spline at $t'$ using De Casteljau's algorithm:
\[ \text{De Casteljau}([p_0, p_{1\cdots O-2}, p_\text{end}], t) = \Delta(x) \]
Because we are using adjacent points from
the embedding for each anchor, we are guaranteed that the endpoints between each spline segment
are continuous. Information is also carried over between segments through the anchor's
representative latent vector, allowing the network to maintain velocity between curves if necessary.

This architecture enforces a guarantee of consistency, regardless of sequence length. It also
naturally allows for disentangling motion between distant time-steps, as while the endpoints of
the spline are fixed, movement in the middle is allowed to be fully independent of other
segments. This architecture is independent from dynamic NeRF, and we propose it for the broader task of generating $C^0$-continuous signals in any domain.

One may question as to why a more complex architecture is necessary as compared to just predicting all the spline points and using only a subset of them. In the case of extremely long sequences, we may expect that the number of spline points increases linearly with the sequence, and for dynamic scene reconstruction we would then have to sample $O(H\times W\times D\times t)$ at each point in space, even though we are only interested in a small subset. With this formulation, we are able to sample sparsely, maintaining a constant memory usage with respect to time.

We do not demonstrate the plausibility of this architecture, due to lack of data known to the
author of long videos with known camera positions, but hope to test it in future work\footnote{If this were to be published, I would demonstrate this architecture, but cannot due to time constraints.}.

\section*{Results}

In order to demonstrate that our method works, we run it on D-NeRF's synthetic
dataset which contains 8 different rendered scenes. These scenes have ground truth camera positions and viewing directions, as well as timestamps. They capture physically plausible movement, so there are no discontinuities or jumps between frames..

\subsection*{Quantitative Results}

The Bezier spline design is able to slightly outperform D-NeRF on their synthetic dataset. This is likely because D-NeRF can predict non-realistic movement, such as jumping from frame-to-frame, or suddenly accelerating. In practice, D-NeRF learns fairly smooth movement, but quantitatively looks different from our methods movement, which is likely due to variations in velocity and acceleration. Using Bezier spline, we enforce that movement is fluid, and thus can better reproduce missing frames. This leads to a small improvement in PSNR, because the dataset contains relatively simple movement, but we expect that in longer sequences or data with large gaps this difference would be more visible.

\subsection*{Qualitative Results}

The difference between our work and D-NeRF can be observed in the difference of flow between scenes. It's clear from Fig.~\ref{fig:dnerf_cmp} that our method more accurately captures squashing and stretching of movement, as the top of the ball is not moving but the bottom of the ball is falling. In contrast, D-NeRF contains approximately equal flow for the entire ball.

The difference between the two is also more clearly seen in videos of reconstruction. $C^0$-NeRF visibly has the effect of 'tweening between views, slowing into stops, while D-NeRF has more less smooth starts and stops. While it's difficult to characterize the plausibility of this movement, future work can measure smoothness of velocity and acceleration and use that as a way to differentiate between the approaches.

\begin{table*}[t]
    \centering
    \begin{tabular}{|c| c|c | c|c | c|c | c|c |}
    \hline
    \textbf{PSNR$^\uparrow$ $|$ MS-SSIM$^\uparrow$} & \multicolumn{2}{c|}{Bouncing Balls} & \multicolumn{2}{c|}{Hellwarrior} & \multicolumn{2}{c|}{Hook} & \multicolumn{2}{c|}{Jumping Jacks} \\
    \hline
    D-NeRF & \textbf{24.646} & \textbf{0.975}
           & 33.314 & 0.968
           & 27.954 & 0.978
           & 27.610 & 0.979 \\
    \hline
    $C^0$-NeRF & 24.251 & 0.971
               & \textbf{33.504} & 0.968
               & \textbf{28.104} & \textbf{0.979}
               & \textbf{27.756} & \textbf{0.981} \\
    \hline
    & \multicolumn{2}{c|}{Lego} & \multicolumn{2}{c|}{Mutant} & \multicolumn{2}{c|}{Standup} & \multicolumn{2}{c|}{T-Rex} \\
    \hline
    D-NeRF & \textbf{23.232} & \textbf{0.940}
           & 28.693 & 0.981
           & 31.307 & 0.989
           & \textbf{25.525} & \textbf{0.975} \\
    \hline
    $C^0$-NeRF & 23.148 & 0.938
               & \textbf{28.979} & \textbf{0.983}
               & \textbf{31.349} & \textbf{0.990}
               & 25.421 & 0.973 \\
    \hline
    \end{tabular}
    \caption{
        Bezier splines are able to recover movement with slightly improved accuracy in dynamic scenes as compared to D-NeRF~\cite{pumarola2020dnerf}. Despite, or maybe because of, the forced prior of continuous movement, we are able to learn a smooth interpolation through each frame. We randomly samples all frames from the start of training, as opposed to D-NeRF which requires slowly introducing new frames, but we also remove the constraint that at time 0 D-NeRF must have no movement. Here, we parametrize $C^0$-NeRF with 6 control points.
    }
\end{table*}

\begin{figure*}
    \includegraphics[width=\textwidth]{dnerf_compare}
    \caption{
        \label{fig:dnerf_cmp}
        Results comparing movement in D-NeRF~\cite{pumarola2020dnerf} versus Bezier-splines for a small timestep. There is substantial difference in the predicted flow, since D-NeRF cannot guarantee the initial frame has no movement. There is also a significant difference in rigidity, which is not easily explained. We believe it may be significantly easier to model low movement with splines, thus there is less of a need for rigidity.
    }
\end{figure*}

\section*{Discussion}

Learned Bezier splines produce coherent movement in objects without explicit regularization, and are able to converge with equal to speed as prior work. Our method demonstrates smooth interpolation, as enforced by the structure of our approach. We are thus able to super-sample between frames at arbitrary resolution while guaranteeing smoothness. In addition, because of the analytic nature of splines, we could perform interesting motion deformations, such as twisting the curves in order to construct novel motion.

Another computational benefit of our approach is that for a given view, we only need to evaluate the deformation MLP \textit{once} to compute the Bezier spline at each point, then we can sample the Bezier spline at an arbitrary number of points. This should allow for efficient single view video reconstruction, but we leave future work for this.

Finally, even without the above optimization, our model runs approximately as fast as NR-NeRF, but can be optimized further, either by introducing voxels, or using other learning approaches such as a different optimization scheme for splines.
\section*{Conclusion}

In conclusion, we devise a new architecture for $C^0$ continuous interpolation, and demonstrate
its efficacy on Dynamic Neural Rendering. Our architecture is able to accurately and quickly
reconstruct scenes, while providing stronger guarantees with a simpler formulation than before.
This hopefully inspires more introduction of classical tools inside of the differentiable rendering pipeline.

\section*{Future Work}

For future work, we plan on applying this architecture to other signals, such as sound for
reconstruction of long duration music. This can be used to generalize over sound, by generating
an embedding from using an encoder-decoder architecture.

Another next step, in the realm of neural rendering, is to encode this using a Plenoxel, so that
rapid training can be done. Since we are storing a fixed number of parameters predicted from a
neural network, this should allow for much more efficient rendering and training, on the order
of 100 times faster, but make it significantly easier to reconstruct dynamic scenes. We note that this is now possible with our formulation, because it is not necessary to query a neural net if the spline is already known, and this can be cached in a sparse voxel structure.

It may also be interesting to explore different formulations of splines, as we only select
Bezier splines due to ease of implementation, but it may be that other splines might be more
numerically stable or have stronger expressivity for reconstruction.


{\small
    \bibliographystyle{ieee_fullname}
    \bibliography{ref}
}

\end{document}